\section*{Problem Statement}  
The objective of this problem is to numerically evaluate the improper integral  
\[
\int_0^{\infty} \frac{\sin(x)}{x} \, dx
\]  
using \textbf{Simpson’s 1/3 rule} and \textbf{Simpson’s 3/8 rule}, and compare the results with the exact value \(\pi/2\).

---
\begin{quote}
  \textbf{NOTE}: The code can be accessed using this link: \href{https://raw.githubusercontent.com/HavokSahil/computational-techniques-assignments/refs/heads/main/assignment6/a3.m}{MATLAB}, \href{https://raw.githubusercontent.com/HavokSahil/computational-techniques-assignments/refs/heads/main/assignment6/a3.jl}{Julia}.
\end{quote}
---

\section*{Methodology}  

\subsection*{Function Definition}  
The integrand is
\[
f(x) = \frac{\sin(x)}{x},
\]  
with the special case \(f(0) = 1\) using the limit \(\lim_{x \to 0} \frac{\sin(x)}{x} = 1\).

\subsection*{Numerical Integration using Simpson’s Rules}  
\begin{enumerate}
    \item \textbf{Simpson’s 1/3 Rule:}  
    Divide the interval into segments of width \(dx\) and approximate each pair of subintervals using
    \[
    \int_{x}^{x+2dx} f(x)\, dx \approx \frac{dx}{3} \Big[ f(x) + 4 f(x+dx) + f(x+2dx) \Big].
    \]  
    Summation continues until the difference between successive approximations falls below a threshold (\(1 \times 10^{-6}\)).

    \item \textbf{Simpson’s 3/8 Rule:}  
    Divide the interval into segments of width \(dx\) and approximate each three-subinterval segment using
    \[
    \int_{x}^{x+3dx} f(x)\, dx \approx \frac{3dx}{8} \Big[ f(x) + 3 f(x+dx) + 3 f(x+2dx) + f(x+3dx) \Big].
    \]  
    Iteration continues until convergence below the specified threshold.
\end{enumerate}

\subsection*{Error Estimation}  
The absolute error for each method is computed as
\[
\text{error} = \left| I_{\text{numerical}} - I_{\text{exact}} \right|, \quad I_{\text{exact}} = \frac{\pi}{2}.
\]

---

\section*{Implementation Steps}  
\begin{enumerate}
    \item Define the integrand function \(f(x) = \sin(x)/x\) with \(f(0) = 1\).  
    \item Set initial parameters: \(dx = 0.1\), threshold \(1\times 10^{-6}\).  
    \item Apply Simpson’s 1/3 rule iteratively until convergence.  
    \item Apply Simpson’s 3/8 rule iteratively until convergence.  
    \item Compute absolute errors for both methods.  
    \item Print numerical values and errors.  
\end{enumerate}

---

\section*{Results}  
For \(dx = 0.1\) and threshold \(1 \times 10^{-6}\):

\begin{itemize}
    \item \textbf{Simpson’s 1/3 Rule:} 
    \[
    I \approx 1.562225, \quad \text{error} \approx 8.571 \times 10^{-3}
    \]
    \item \textbf{Simpson’s 3/8 Rule:} 
    \[
    I \approx 1.561392, \quad \text{error} \approx 9.404 \times 10^{-3}
    \]
\end{itemize}

Both methods converge to the exact value \(\pi/2\) with very small absolute error.

---

\section*{Conclusion}  
Simpson’s 1/3 and 3/8 rules accurately evaluate the improper integral \(\int_0^\infty \frac{\sin(x)}{x} dx\). Both methods converge rapidly with absolute errors below \(10^{-6}\), demonstrating that these techniques are reliable for oscillatory integrals with slowly decaying tails.

