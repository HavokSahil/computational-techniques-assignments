\section*{Problem Statement}  
The objective of this problem is to solve the \textbf{particle in a box} quantum system numerically using the \textbf{central difference method}, compare it with the analytical solution, and normalize both wave functions using numerical integration.
---

\begin{quote}
  \textbf{NOTE}: The code can be accessed using this link: \href{https://raw.githubusercontent.com/HavokSahil/computational-techniques-assignments/refs/heads/main/assignment6/a2.m}{MATLAB}, \href{https://raw.githubusercontent.com/HavokSahil/computational-techniques-assignments/refs/heads/main/assignment6/a2.jl}{Julia}.
\end{quote}

---

\section*{Methodology}  

\subsection*{Discretization and Finite Difference Scheme}  
1. The spatial domain \([x_0, x_N] = [0, 4]\) is discretized into \(N = 300\) intervals with spacing
\[
dx = \frac{x_N - x_0}{N}.
\]  
2. The \textbf{internal grid points} (excluding boundaries) are used to construct a tridiagonal matrix representing the second derivative in the Schrödinger equation:
\[
A = \begin{bmatrix}
2 & -1 & 0 & \dots & 0 \\
-1 & 2 & -1 & \dots & 0 \\
0 & -1 & 2 & \dots & 0 \\
\vdots & \vdots & \vdots & \ddots & -1 \\
0 & 0 & 0 & -1 & 2
\end{bmatrix}.
\]

\subsection*{Eigenvalue Computation (Inverse Iteration)}  
- A random initial wavefunction guess is iteratively updated using the inverse of \(A\) to find the smallest eigenvalue and corresponding eigenvector.  
- The eigenvalue is computed as
\[
\lambda_{\text{min}} = \frac{\mathbf{wf}^T A \mathbf{wf}}{\mathbf{wf}^T \mathbf{wf}}.
\]

\subsection*{Wave Function Normalization}  
- The numerical wave function is extended with zero boundary conditions at \(x_0\) and \(x_N\).  
- Normalization is performed using the trapezoidal rule:
\[
\int_0^L |\psi(x)|^2 dx \approx \sum_{i=1}^{N} \frac{dx}{2} \Big[ |\psi(x_i)|^2 + |\psi(x_{i+1})|^2 \Big],
\]
with scaling factor \(a = \sqrt{1/I}\).  

\subsection*{Analytical Solution}  
- The analytical solution for the lowest energy state of a particle in a box of length \(L=4\) is
\[
\psi_{\text{analytical}}(x) = \sin\left(\frac{\pi x}{L}\right),
\]  
which is also normalized numerically using the trapezoidal rule.

---

\section*{Implementation Steps}  
\begin{enumerate}
    \item Define the spatial grid and grid spacing.
    \item Construct the tridiagonal matrix representing the central difference approximation of the second derivative.
    \item Apply inverse iteration to compute the smallest eigenvalue and eigenvector.
    \item Extend the wave function with zero boundary conditions.
    \item Normalize the numerical wave function using numerical integration.
    \item Compute and normalize the analytical wave function.
    \item Compare and visualize numerical vs analytical wave functions.
\end{enumerate}

---

\section*{Results}  
\begin{enumerate}
    \item The numerical wave function matches closely with the analytical solution \(\psi(x) = \sin(\pi x / L)\).  
    \item Both wave functions are normalized (\(\int_0^L |\psi(x)|^2 dx = 1\)).  
\end{enumerate}

\begin{figure}[h!]
  \centering
  \includegraphics[width=0.9\textwidth]{a2.png}
  \caption{Comparison of numerical (red solid) and analytical (black dashed) wave functions for the particle in a box.}
\end{figure}

---

\section*{Conclusion}  
The central difference method successfully computes the numerical wave function and lowest eigenvalue for a particle in a box. Normalization using numerical integration ensures the wave function has unit probability. The numerical solution agrees closely with the analytical \(\sin(\pi x / L)\) solution, demonstrating the accuracy and reliability of the finite difference approach for solving simple quantum systems.

