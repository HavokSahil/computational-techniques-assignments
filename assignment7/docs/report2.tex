\section*{Problem Statement}  
The objective of this problem is to \textbf{numerically integrate} the function
\[
f(x) = \frac{1}{x+1}
\]
over the interval \(x \in [0, 1]\) using the \textbf{4-point Gauss-Legendre quadrature} method.  
Since Gauss-Legendre nodes are defined over \([-1,1]\), a \textbf{change of variables} is applied to transform the interval \([0,1]\) to \([-1,1]\).

\section*{Methodology}  

\subsection*{Change of Variables}  
The interval \([0,1]\) is mapped to \([-1,1]\) using
\[
x = \frac{b-a}{2} t + \frac{b+a}{2} = \frac{t+1}{2}, \quad dx = \frac{1}{2} dt
\]  
where \(t \in [-1,1]\) represents the standard Gauss-Legendre variable.  
The transformed integral becomes:
\[
\int_0^1 \frac{1}{x+1} \, dx = \frac{1}{2} \sum_{i=0}^{3} w_i \, f\Big(\frac{t_i+1}{2}\Big)
= \sum_{i=0}^{3} w_i \, \frac{1}{t_i + 3} 
\]

\subsection*{Gauss-Legendre Quadrature (4-point)}  
The 4-point Gauss-Legendre quadrature approximates the integral as a weighted sum of function values at specific nodes:

\[
I \approx \sum_{i=0}^{3} w_i \, f(x_i)
\]

with precomputed nodes and weights:

\[
\begin{aligned}
x_0 &= -0.86114, & w_0 &= 0.34785 \\
x_1 &= -0.33998, & w_1 &= 0.65215 \\
x_2 &=  0.33998, & w_2 &= 0.65215 \\
x_3 &=  0.86114, & w_3 &= 0.34785
\end{aligned}
\]

\subsection*{Implementation Steps}  
\begin{enumerate}
    \item Define the mapped function \(f(t) = \frac{1}{t+3}\).
    \item Specify the 4 Gauss-Legendre nodes and weights.
    \item Compute the weighted sum \(I = \sum_{i=0}^{3} w_i f(t_i)\) to approximate the integral.
    \item Display the result.
\end{enumerate}

\section*{Results}  
The numerical integration using 4-point Gauss quadrature gives:

\begin{itemize}
    \item \textbf{Integral Value:} \(I \approx 0.693146\)
\end{itemize}

\section*{Conclusion}  
The 4-point Gauss-Legendre quadrature accurately approximates the integral of
\(f(x) = \frac{1}{x+1}\) over \([0, 1]\).  
Using the change of variables, standard Gauss-Legendre nodes and weights can be applied to any finite interval, demonstrating the flexibility and efficiency of the Gauss quadrature method.

