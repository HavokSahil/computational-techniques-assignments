\section*{Problem Statement}  
The objective of this problem is to \textbf{numerically integrate} the function
\[
f(x) = \frac{8(x+3)}{16 + (x+3)^4}
\]
over the interval \(x \in [-1, 1]\) using the \textbf{4-point Gauss-Legendre quadrature} method.

\section*{Methodology}  

\subsection*{Gauss-Legendre Quadrature (4-point)}  
The Gauss-Legendre quadrature method approximates the definite integral
\[
\int_{-1}^{1} f(x) \, dx
\]
as a weighted sum of function values at specific nodes:

\[
I \approx \sum_{i=0}^{3} w_i \, f(x_i)
\]

where \(x_i\) are the Gauss-Legendre nodes and \(w_i\) are the corresponding weights.  
For 4-point quadrature, the nodes and weights are precomputed as:

\[
\begin{aligned}
x_0 &= -0.86114, & w_0 &= 0.34785 \\
x_1 &= -0.33998, & w_1 &= 0.65215 \\
x_2 &=  0.33998, & w_2 &= 0.65215 \\
x_3 &=  0.86114, & w_3 &= 0.34785
\end{aligned}
\]

\subsection*{Implementation Steps}  
\begin{enumerate}
    \item Define the function \(f(x) = \frac{8(x+3)}{16 + (x+3)^4}\).
    \item Specify the 4 Gauss-Legendre nodes and weights.
    \item Compute the weighted sum \(I = \sum_{i=0}^{3} w_i f(x_i)\) to approximate the integral.
    \item Display the result.
\end{enumerate}

\section*{Results}  
The numerical integration using 4-point Gauss quadrature gives:

\begin{itemize}
    \item \textbf{Integral Value:} \(I \approx 0.540410\)
\end{itemize}

\section*{Conclusion}  
The 4-point Gauss-Legendre quadrature accurately approximates the integral of
\(f(x) = \frac{8(x+3)}{16 + (x+3)^4}\) over \([-1, 1]\).  
For smooth functions, Gauss quadrature achieves high accuracy with very few nodes compared to methods like the trapezoidal rule.

