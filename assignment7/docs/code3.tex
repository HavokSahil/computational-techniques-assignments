\begin{lstlisting}[language=Matlab,basicstyle=\ttfamily\small,breaklines=true]
%% Program to compute the expansion of the given
%% function in terms of legendre polynomials
%%
%% Author: Sahil Raj
%% Assignment 7 Problem 3

clc; clear all;

% Load data
T = dlmread("Guitar\_Theta\_R.txt", "\t");
A = T(:,1);   % angles in degrees
D = T(:,2);   % distances

theta = deg2rad(A);
x = cos(theta);

% Gauss-Legendre quadrature nodes and weights (10-point)
xi = [
  -0.9739065285;
  -0.8650633666;
  -0.6794095682;
  -0.4333953941;
  -0.1488743389;
   0.1488743389;
   0.4333953941;
   0.6794095682;
   0.8650633666;
   0.9739065285
];
wi = [
   0.0666713443;
   0.1494513491;
   0.2190863625;
   0.2692667193;
   0.2955242247;
   0.2955242247;
   0.2692667193;
   0.2190863625;
   0.1494513491;
   0.0666713443
];

% Interpolation function R(x)
function Rval = computeR(xval, xdata, Ddata)
    % Interpolate using pchip for stability
    Rval = interp1(xdata, Ddata, xval, 'pchip', 'extrap');
endfunction

% Legendre polynomial using recurrence
function Pl = legendre\_numeric(l, x)
    if l == 0
        Pl = 1;
        return;
    elseif l == 1
        Pl = x;
        return;
    end
    P0 = 1;
    P1 = x;
    for n = 1:l-1
        Pn1 = ((2*n+1) * x .* P1 - n * P0)/(n+1);
        P0 = P1;
        P1 = Pn1;
    end
    Pl = P1;
endfunction

% Compute coefficients
maxL = 15;      % maximum Legendre order
C = zeros(maxL+1,1);
for l = 0:maxL
    sumval = 0.0;
    for i = 1:length(xi)
        Rval = computeR(xi(i), x, D);
        Pl = legendre\_numeric(l, xi(i));
        sumval = sumval + wi(i) * Rval * Pl;
    end
    C(l+1) = (2*l+1)/2 * sumval;   % normalization factor for P\_l
end

% Reconstruct R(theta) from Legendre expansion
theta\_plot = linspace(min(theta), max(theta), 300);
x\_plot = cos(theta\_plot);
R\_rec = zeros(size(x\_plot));
for l = 0:maxL
    R\_rec = R\_rec + C(l+1) * legendre\_numeric(l, x\_plot);
end

% Plot
figure;
plot(rad2deg(theta\_plot), R\_rec, 'r-', 'LineWidth', 2); hold on;
plot(A, D, 'bo');
xlabel('Angle (degrees)');
ylabel('R(\theta)');
title('Radial Function and Legendre Expansion Approximation');
legend('Reconstructed R(\theta)','Original Data');
grid on;

\end{lstlisting}
