\begin{lstlisting}[language=Matlab,basicstyle=\ttfamily\small,breaklines=true]

ng = 4;
nodes = [
    -sqrt((3/7) - (2/7)*sqrt(6/5));
    -sqrt((3/7) + (2/7)*sqrt(6/5));
     sqrt((3/7) + (2/7)*sqrt(6/5));
     sqrt((3/7) - (2/7)*sqrt(6/5))
];

weights = [
    (18 - sqrt(30)) / 36;
    (18 + sqrt(30)) / 36;
    (18 + sqrt(30)) / 36;
    (18 - sqrt(30)) / 36
];


%% Physical constants
rho0 = 1.0;
eps0 = 1.0;
a    = 1.0;


%% Integrand: p\^2 * e\^{-(R\^2 p\^2)/a\^2}
function y = fun(p, R, a)
    y = exp(-(R*R)*(p*p)/(a*a)) * (p*p);
endfunction


function E = computeE(R, nodes, weights, rho0, eps0, a)
    E = (rho0 * R) / (2 * eps0);

    I = 0;
    for i = 1:length(nodes)
        I += weights(i) * fun(nodes(i), R, a);
    end

    E *= I;
endfunction


%% Total charge using 2-pt Gauss-Hermite rule
x  = [-1/sqrt(2); 1/sqrt(2)];
wH = [sqrt(pi)/2; sqrt(pi)/2];

Q = 2*pi*a\^3*rho0 * (wH(1)*x(1)\^2 + wH(2)*x(2)\^2);
fprintf("Total charge in space: %.6f\n", Q);


%% Compute E(R) over range
Rs = 0.1:0.01:10;
Es = arrayfun(@(R) computeE(R, nodes, weights, rho0, eps0, a), Rs);


%% PLOT
figure;
plot(Rs, Es, "linewidth", 2);
grid on;
xlabel("R");
ylabel("E(R)");
title("Electric Field vs Radius Using 4-Point Gauss-Legendre Quadrature");
\end{lstlisting}
