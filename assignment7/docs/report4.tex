\section*{Problem Statement}
The goal of this experiment is to \textbf{verify the orthogonality of Legendre polynomials} using \textbf{3-point Gauss--Legendre quadrature}.  
The known orthogonality condition is:
\[
\int_{-1}^{1} P_i(x)\,P_j(x)\,dx = 
\begin{cases}
\dfrac{2}{2i+1}, & \text{if } i=j, \\[6pt]
0, & \text{if } i \neq j.
\end{cases}
\]

A numerical quadrature scheme is used to approximate these integrals and compare them against the theoretical values.

\section*{Methodology}

\subsection*{Gauss--Legendre Quadrature}
A \textbf{3-point Gauss--Legendre rule} is used.  
It integrates exactly all polynomials up to degree:
\[
2n - 1 = 5.
\]
Thus, the integral of \(P_i P_j\) is accurate only when:
\[
i + j \le 5.
\]

The nodes and weights are:
\[
x = \{-\sqrt{3/5},\ 0,\ \sqrt{3/5}\},
\quad
w = \left\{\frac{5}{9},\ \frac{8}{9},\ \frac{5}{9}\right\}.
\]

\subsection*{Legendre Polynomial Evaluation}
Legendre polynomials up to order 5 are explicitly defined:
\[
P_0 = 1,\quad
P_1 = x,\quad
P_2 = \frac{1}{2}(3x^2 - 1),\quad
P_3 = \frac{1}{2}(5x^3 - 3x),
\]
\[
P_4 = \frac{1}{8}(35x^4 - 30x^2 + 3),\quad
P_5 = \frac{1}{8}(63x^5 - 70x^3 + 15x).
\]

These expressions form the integrand:
\[
f_{ij}(x) = P_i(x)\,P_j(x).
\]

\subsection*{Numerical Integration}
For each pair \((i,j)\) with \(0 \le i,j \le 5\), the integral is approximated as:
\[
I_{ij} \approx \sum_{k=1}^{3} w_k\, P_i(x_k)\,P_j(x_k).
\]

To compare with the analytical orthogonality formula, the diagonal entries are normalized by:
\[
\frac{2i+1}{2},
\]
so the expected normalized matrix becomes:
\[
M_{ij} =
\begin{cases}
1, & i=j,\\
0, & i\ne j.
\end{cases}
\]

Only entries with \(i+j \le 5\) are meaningful because of the degree limitation of the quadrature rule.

\section*{Results}

\subsection*{Orthogonality Matrix}
The computed matrix shows:
\begin{itemize}
    \item diagonal entries very close to 1,
    \item off-diagonal entries near 0,
    \item expected inaccuracies whenever \(i+j > 5\), since the 3-point rule cannot integrate such products exactly.
\end{itemize}

\begin{figure}[h!]
\centering
\includegraphics[width=0.8\textwidth]{matrix.jpg}
\caption{Computed orthogonality matrix using 3-point Gauss--Legendre quadrature.}
\end{figure}

\subsection*{Legendre Polynomial Plots}
The Legendre polynomials \(P_0\) through \(P_5\) are plotted over \([-1,1]\), showing the typical oscillatory structure and symmetry.

\begin{figure}[h!]
\centering
\includegraphics[width=0.8\textwidth]{a4.jpg}
\caption{Plots of Legendre polynomials \(P_0(x)\) to \(P_5(x)\).}
\end{figure}

\section*{Conclusion}
This experiment confirms the expected orthogonality of Legendre polynomials using a 3-point Gauss--Legendre quadrature rule.  
The results match theoretical values for all \((i,j)\) such that \(i+j \le 5\).  
This verifies:
\begin{itemize}
    \item the correctness of the Legendre polynomial formulas,
    \item the reliability of Gauss--Legendre quadrature,
    \item and the orthogonality structure of the polynomial family.
\end{itemize}

The visualizations further support these findings.
