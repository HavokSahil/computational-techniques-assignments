\begin{lstlisting}[language=Matlab,basicstyle=\ttfamily\small,breaklines=true]
clc; clear all;
%%  Verification of the Orthogonality of Legendre Polynomials
%%  Using 3-Point Gauss-Legendre Quadrature
%%
%%  The 3-point quadrature formula is exact for polynomials of
%%  degree <= 5. Therefore, the integral of P\_i * P\_j is reliable
%%  only when i + j <= 5.
%%
%%  Author : Sahil Raj
%%  Course : Assignment 7 - Problem 4

N = 5;                 % Maximum order to test

%%  Quadrature Data (3-Point Gauss-Legendre)
%%  These weights and nodes integrate exactly up to degree 5.
global nP;    nP    = 3;
global lambda; lambda = [ 5/9; 8/9; 5/9 ];
global nodeps; nodeps = [ -sqrt(3/5); 0; sqrt(3/5) ];


%%  Custom Legendre Polynomials P\_0 ... P\_5
function y = LegendreCustom(n, x)
    switch n
        case 0, y = 1.0;
        case 1, y = x;
        case 2, y = (3*x*x - 1)/2;
        case 3, y = (5*x\^3 - 3*x)/2;
        case 4, y = (35*x\^4 - 30*x\^2 + 3)/8;
        case 5, y = (63*x\^5 - 70*x\^3 + 15*x)/8;
        otherwise
            error("LegendreCustom: order not implemented.");
    end
endfunction


%%  Gauss Quadrature Integrator
function I = gaussQuad(fn)
    global nP lambda nodeps;
    I = 0.0;
    for i = 1:nP
        I += lambda(i) * fn(nodeps(i));
    end
endfunction


%%  Compute Orthogonality Matrix
intmatrix = zeros(N+1, N+1);

for i = 0:N
    for j = 0:N
        fn = @(x) LegendreCustom(i,x) * LegendreCustom(j,x);
        I = gaussQuad(fn);

        % Apply normalization for diagonal entries:
        if i == j
            I *= (2*i + 1) / 2;
        endif

        intmatrix(i+1, j+1) = I;
    endfor
endfor

%% Display matrix
disp("Orthogonality Matrix:");
disp(intmatrix);


%%  PLOTTING SECTION

% Plot Legendre polynomials P0..P5
x = linspace(-1, 1, 400);
figure(1); hold on; grid on;
title("Legendre Polynomials P\_0 to P\_5");
xlabel("x"); ylabel("P\_n(x)");

for n = 0:5
    plot(x, arrayfun(@(t) LegendreCustom(n,t), x));
endfor
legend("P\_0","P\_1","P\_2","P\_3","P\_4","P\_5");


% Heatmap-like plot of orthogonality matrix
figure(2);
imagesc(intmatrix);
title("Orthogonality Matrix (Computed)");
xlabel("j"); ylabel("i");
colorbar();
set(gca, "XTick", 1:N+1, "YTick", 1:N+1);

\end{lstlisting}
