\section*{Problem Statement} 
The objective of this problem is to solve a system of linear equations using the Gaussian elimination method with partial pivoting. Partial pivoting improves numerical stability by reducing round-off errors, especially when pivot elements are small or zero.

\section*{Methodology} 
Gaussian elimination with partial pivoting proceeds in two phases:

\begin{enumerate}
  \item \textbf{Pivoting:} At each step, reorder the rows such that the largest (by absolute value) candidate for the pivot element is placed on the diagonal. This avoids division by small numbers.
  \item \textbf{Forward elimination:} Use the pivot row to eliminate entries below the pivot, reducing the system to an upper-triangular form.
  \item \textbf{Back-substitution:} Solve for the variables starting from the last equation upwards:
  \[
  x_i = \frac{A_{i,n+1} - \sum_{j=i+1}^{n} A_{i,j}x_j}{A_{i,i}}.
  \]
\end{enumerate}

\subsection*{Pseudo-code}
\begin{enumerate}
  \item Store $M$ and $R$.
  \item Form augmented matrix $A = [M|R]$.
  \item For each pivot index $p = 1 \to n-1$:
  \begin{itemize}
    \item Swap rows to place the largest pivot in row $p$ (partial pivoting).
    \item For each row $r > p$, eliminate $A(r,p)$ using the pivot row.
  \end{itemize}
  \item Apply back-substitution to compute the solution vector.
\end{enumerate}

\section*{Results} 
The given system of equations is:
\[
\begin{aligned}
0x + 2y + z &= -8, \\
x - 2y - 2z &= 0, \\
- x + y + 2z &= 3.
\end{aligned}
\]

The augmented matrix is:
\[
A =
\begin{bmatrix}
0 & 2 & 1 & | & -8 \\
1 & -2 & -2 & | & 0 \\
-1 & 1 & 2 & | & 3
\end{bmatrix}.
\]

After applying partial pivoting, the sorted augmented matrix becomes:
\[
A =
\begin{bmatrix}
1 & -2 & -2 & | & 0 \\
0 & 2 & 1 & | & -8 \\
-1 & 1 & 2 & | & 3
\end{bmatrix}.
\]

Forward elimination reduces this system to upper-triangular form, and back-substitution yields:
\[
x = -10.00, \quad y = -3.00, \quad z = -2.00.
\]

\section*{Conclusion} 
Gaussian elimination with partial pivoting was successfully implemented to solve the given system of equations. The pivoting step ensured stability by preventing division by zero (since the original first pivot was zero). The computed solution confirms that partial pivoting is an essential enhancement to the standard Gaussian elimination method when working with arbitrary systems.

