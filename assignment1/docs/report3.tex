\section*{Problem Statement} 
The objective of this problem is to solve a system of linear equations using the Gauss–Jordan elimination method. Unlike Gaussian elimination, which reduces the system to an upper-triangular form followed by back-substitution, Gauss–Jordan elimination directly reduces the augmented matrix to a diagonal (row-reduced echelon) form. This makes the solution values explicit without requiring further substitution.

\section*{Methodology} 
The Gauss–Jordan method transforms the augmented matrix $[M|R]$ into reduced row-echelon form (RREF). The steps are:

\begin{enumerate}
  \item Form the augmented matrix $A = [M|R]$.
  \item For each pivot column $p = 1 \to n$:
  \begin{itemize}
    \item Identify the pivot element $A(p,p)$.
    \item For each row $r \neq p$, eliminate $A(r,p)$ by subtracting a multiple of the pivot row.
  \end{itemize}
  \item Once all off-diagonal elements are zero, divide each row by its pivot element so that the coefficient matrix becomes the identity.
  \item The last column of the reduced matrix contains the solution vector.
\end{enumerate}

\subsection*{Pseudo-code}
\begin{enumerate}
  \item Input coefficient matrix $M$ and right-hand side vector $R$.
  \item Construct augmented matrix $A = [M|R]$.
  \item For each pivot index $p = 1 \to n$:
  \begin{itemize}
    \item For each row $r \neq p$, set 
    \[
    A(r,:) = A(r,:) - \frac{A(r,p)}{A(p,p)} \cdot A(p,:).
    \]
  \end{itemize}
  \item Compute solution as $x_i = A(i,n+1)/A(i,i)$.
  \item Output the solution vector.
\end{enumerate}

\section*{Results} 
The given system of equations is:
\[
\begin{aligned}
-76x + 25y + 50z &= -10, \\
25x - 56y + 1z &= 0, \\
50x + y - 106z &= 0.
\end{aligned}
\]

The augmented matrix is:
\[
A =
\begin{bmatrix}
-76 & 25 & 50 & | & -10 \\
25 & -56 & 1 & | & 0 \\
50 & 1 & -106 & | & 0
\end{bmatrix}.
\]

Applying Gauss–Jordan elimination reduces the matrix to diagonal form, from which the solution is read directly:
\[
x = 0.2449, \quad y = 0.1114, \quad z = 0.1166.
\]

\section*{Conclusion} 
The Gauss–Jordan elimination method was applied to solve the system of three equations. Unlike standard Gaussian elimination, no back-substitution was necessary since the algorithm produced a diagonal system. The obtained values of $x$, $y$, and $z$ represent the loop currents in the given electrical circuit (clockwise in loops 1, 2, and 3 respectively), validating the method’s application to circuit analysis problems.

