\section*{Jacobi Diagonalization of a 3-Level Quantum Hamiltonian}

\subsection*{Problem Statement}

Consider a quantum system with three basis states:
\[
|1\rangle,\; |2\rangle,\; |3\rangle,
\]
with unperturbed energies \(E_1, E_2, E_3\). In the presence of coupling \(V\), the Hamiltonian becomes:
\[
H = H_0 + V,
\]
introducing mixing between the basis states.  
For the given atomic system, the Hamiltonian matrix is:
\[
H =
\begin{bmatrix}
1.0 & 0.2 & 0.1 \\
0.2 & 2.0 & 0.3 \\
0.1 & 0.3 & 3.0
\end{bmatrix}.
\]

Our goal is to diagonalize this Hamiltonian using the \textbf{Jacobi rotation method} and obtain:
\begin{itemize}
    \item the \textbf{eigenvalues} (physical energy levels),
    \item the \textbf{eigenvectors} (mixing of the basis states).
\end{itemize}

\subsection*{Methodology}

The Hamiltonian is real and symmetric, so Jacobi diagonalization can be used.  
The procedure is:

\begin{enumerate}
    \item Identify the largest off-diagonal component \(H_{pq}\).
    \item Compute the rotation angle:
    \[
    \theta = 
    \frac{1}{2}
    \tan^{-1}\left(\frac{2 H_{pq}}{H_{pp} - H_{qq}}\right).
    \]
    \item Construct a planar Givens rotation matrix \(R\) mixing rows/columns \(p\) and \(q\).
    \item Update the Hamiltonian:
    \[
    H \leftarrow RHR^{T}.
    \]
    \item Accumulate eigenvectors:
    \[
    Q \leftarrow Q R.
    \]
\end{enumerate}

Iterations stop when all off-diagonal elements fall below threshold.

\subsection*{Results}

After convergence, the diagonal matrix \(D\) and eigenvector matrix \(Q\) are obtained as:

\[
\textbf{Eigenvalues:}\qquad
\lambda = (0.9606,\; 1.9455,\; 3.0939).
\]

\[
\textbf{Eigenvectors (columns of $Q$):}
\qquad
Q =
\begin{bmatrix}
0.982891 & -0.172578 & -0.064360 \\
0.145984 & \;\,0.942982 & -0.299120 \\
0.112312 & \;\,0.284607 & \;\,0.952042
\end{bmatrix}.
\]

\subsection*{Physical Interpretation}

Each column of \(Q\) is a normalized eigenvector:
\[
|\psi_i\rangle
= c_{1i}|1\rangle + c_{2i}|2\rangle + c_{3i}|3\rangle,
\]
describing how the original basis states mix due to the off-diagonal couplings.

\begin{itemize}
    \item The lowest energy eigenstate (0.9606) is dominated by \(|1\rangle\) with small admixtures of \(|2\rangle, |3\rangle\).
    \item The middle state (1.9455) is mostly \(|2\rangle\), but significantly mixed.
    \item The highest state (3.0939) is largely \(|3\rangle\), with small contributions from the first two.
\end{itemize}

\subsection*{Conclusion}

The Jacobi diagonalization successfully produced the eigenvalues and eigenvectors of the coupled 3-state Hamiltonian. The results clearly show mixing between the basis states due to the off-diagonal couplings. The eigenvectors form an orthonormal basis, representing the physical stationary states of the system.


