\section*{Problem Statement}
The objective is to solve the \textbf{Schrödinger equation} for the ground state of a particle in a harmonic potential using the \textbf{central difference iterative method}, and then \textbf{normalize} the wavefunction using the \textbf{2-point Gauss-Hermite quadrature}.

The time-independent Schrödinger equation for a one-dimensional harmonic oscillator is:
\[
-\frac{\hbar^2}{2 m} \frac{d^2 \psi(x)}{dx^2} + \frac{1}{2} m \omega^2 x^2 \psi(x) = E \psi(x)
\]

\section*{Methodology}

\subsection*{Central Difference Iterative Formula}
The second derivative is approximated using central differences:
\[
\psi_{i+1} = \phi(dx, i)\, \psi_i - \psi_{i-1},
\]
where \(\phi(dx, i) = 2 + (dx^4) j^2 - dx^2\), \(dx\) is the step size, and \(j\) indexes the spatial grid.  

The iterative scheme computes the wavefunction on a discrete grid \(x \in [0, X_{\max}]\).

\subsection*{Normalization using Gauss-Hermite 2-point Quadrature}
For a function weighted by \(e^{-x^2}\), the 2-point Gauss-Hermite quadrature approximates the integral as:
\[
\int_{-\infty}^{\infty} f(x) e^{-x^2} dx \approx \sum_{i=0}^{1} w_i f(x_i),
\]
with nodes and weights:

\[
\begin{aligned}
x_0 &= -\frac{1}{\sqrt{2}}, & w_0 &= \frac{\sqrt{\pi}}{2} \\
x_1 &=  \frac{1}{\sqrt{2}}, & w_1 &= \frac{\sqrt{\pi}}{2}
\end{aligned}
\]

Since the ground-state wavefunction is even, the integral can be simplified and multiplied by 2.

\newpage
\subsection*{Calculation}
\begin{figure}[h!]
\centering
\includegraphics[height=0.65\textheight]{a101.jpg}
\end{figure}
\newpage

\subsection*{Implementation Steps}
\begin{enumerate}
    \item Define constants: \(m, \hbar, \omega, dx\).
    \item Initialize the wavefunction with arbitrary scale at the first two grid points.
    \item Iterate using the central difference formula to compute the wavefunction over the spatial grid.
    \item Normalize the wavefunction using 2-point Gauss-Hermite quadrature.
    \item Plot the normalized wavefunction.
\end{enumerate}

\section*{Code}
\begin{lstlisting}[language=Matlab, caption={Ground State Wavefunction for Harmonic Oscillator using Central Difference and 2-point Gauss-Hermite}]
clc; clear all;

% Constants
m = 1.0;
h = 1.0;
w = 1.0;
dx = 0.01;

function phi = compphi(dx, j)
  phi = 2 + power(dx, 4) * j * j - dx*dx;
endfunction

% Initial values to arbitrary scale
psi0even = 1.0;
psi1even = 1.0;

Xmax = 2;
Xs = 0:dx:Xmax;
N = length(Xs);

psieven = zeros(N, 1);
psieven(1) = psi0even;
psieven(2) = psi1even;

% Compute wavefunction using central difference iterative formula
for i = 3:N
  psieven(i) = compphi(dx, i) * psieven(i-1) - psieven(i-2);
endfor

% 2-point Gauss-Hermite quadrature for normalization
weights = [sqrt(pi)/2; sqrt(pi)/2];
nodes = [-1/sqrt(2); 1/sqrt(2)];

index = floor(nodes(2) / dx);
fx = psieven(index);
I = 2 * fx * weights(2);

psieven = psieven / I;

fprintf("The normalization constant was: %f\n", I);

% Plotting
figure('Color','w');
plot(Xs, psieven, 'LineWidth', 2);
grid on;
xlabel('x','FontSize',12,'FontWeight','bold');
ylabel('\psi(x)','FontSize',12,'FontWeight','bold');
title('Wavefunction','FontSize',14,'FontWeight','bold');
xlim([0 Xmax]);
ylim([min(psieven)-0.1, max(psieven)+0.1]);
set(gca, 'FontSize', 12, 'LineWidth', 1.2);
\end{lstlisting}

\section*{Results}
The central difference iterative method successfully generates the discrete wavefunction for the ground state.  

After normalization using the 2-point Gauss-Hermite method, the wavefunction satisfies the normalization condition:
\[
\int_{-\infty}^{\infty} |\psi(x)|^2 dx = 1
\]

\begin{figure}[h!]
\centering
\includegraphics[width=0.7\textwidth]{a102.jpg}
\caption{Normalized ground-state wavefunction for the harmonic oscillator. Replace this placeholder with the actual plot image generated from MATLAB.}
\end{figure}

\section*{Conclusion}
The central difference iterative formula provides an efficient numerical method to solve the Schrödinger equation for the harmonic oscillator.  

Normalization using 2-point Gauss-Hermite quadrature ensures the wavefunction has unit probability. The method is straightforward and produces accurate results for the ground state, capturing the characteristic Gaussian profile of the harmonic oscillator.

