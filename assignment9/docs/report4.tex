\section*{Problem Statement}

The goal of this problem is to numerically solve the time-dependent Schrödinger equation (TDSE)
\[
i \frac{\partial \psi}{\partial t}
=
-\frac{1}{2}\frac{\partial^2 \psi}{\partial x^2}
+ V(x,t)\,\psi(x,t),
\]
for a particle placed in a time-dependent ``breathing'' harmonic potential
\[
V(x,t) = \frac12\,k(t)\,x^2,
\qquad 
k(t) = \cos^2\!\left(\frac{2\pi t}{T}\right).
\]

The initial wavefunction is chosen as the ground-state of a harmonic oscillator:
\[
\psi(x,0) = \left(\frac{1}{\pi}\right)^{1/4} e^{-x^2/2}.
\]

The objective is to evolve this wavefunction forward in time and study how the probability density
\(|\psi(x,t)|^2\) behaves as the trap strength oscillates.

\section*{Methodology}

\subsection*{Discretization}

The spatial domain is taken as
\[
x\in[-5,5], \qquad N_x = 200,
\]
and the spacing is
\[
\Delta x = \frac{x_{\max}-x_{\min}}{N_x-1}.
\]

The simulation runs for
\[
\Delta t = 0.0005, \qquad N_t = 2000
\]
time steps.

\subsection*{Numerical Update}

The second derivative is approximated using the usual central finite-difference expression,
\[
\frac{\partial^2\psi}{\partial x^2}
\approx
\frac{\psi_{i+1}-2\psi_i+\psi_{i-1}}{\Delta x^2}.
\]

At each time step, the potential is updated according to the breathing function
\[
k(t) = \cos^2\!\left(\frac{2\pi t}{T}\right),
\]
and the wavefunction is advanced using the simple explicit update formula implemented in the code.  
Boundary conditions
\[
\psi(-5,t)=\psi(5,t)=0
\]
are applied at every step.

The quantity \(|\psi(x,t)|^2\) is stored for visualization, and a heatmap of its logarithm is plotted.

\section*{Results}

The expected behaviour was a clear breathing pattern, where the wavefunction periodically expands and contracts as the trap oscillates.  
However, during the simulation the following was observed:

\begin{itemize}
    \item The values of \(|\psi|^2\) steadily increase with time.
    \item The gradient in the heatmap becomes stronger as time progresses.
    \item The breathing pattern is not clearly visible in the output.
\end{itemize}

A sample heatmap produced by the simulation is shown below.

\begin{figure}[h!]
    \centering
    \includegraphics[width=0.85\textwidth]{a4.jpg}
    \caption{Heatmap of $\log_{10}(|\psi(x,t)|^2)$ obtained from the simulation.}
\end{figure}

\section*{Current Issue and Instructor Feedback Request}

At the moment, the simulation does not show the breathing motion that was expected from the physical setup.  
Instead, the magnitude seems to grow over time, and I am not sure where the issue lies in the numerical procedure.

Since I am still learning these methods, I am unsure whether the problem comes from the time step, the discretization, or something in the update formula.

\textbf{I would like to request guidance from the instructor on how to correct this and obtain the intended breathing behaviour.}

\section*{Conclusion}

A numerical simulation of a particle in a breathing harmonic potential was performed, but the expected oscillatory behaviour was not clearly observed.  
The results show growth in the wavefunction amplitude, and further clarification is needed regarding how to stabilize or correct the procedure.

