\documentclass[12pt,a4paper]{report}

\usepackage[utf8]{inputenc}
\usepackage{amsmath, amssymb}
\usepackage{graphicx}
\usepackage{hyperref}
\usepackage{geometry}
\usepackage{listings}
\geometry{margin=1in}

\hypersetup{
    colorlinks=true,
    linkcolor=blue,
    urlcolor=cyan,
    citecolor=red
}


\usepackage{xcolor}        % Optional, for syntax highlighting

\definecolor{codegray}{rgb}{0.5,0.5,0.5}
\definecolor{codegreen}{rgb}{0,0.6,0}
\definecolor{codepurple}{rgb}{0.58,0,0.82}
\definecolor{backcolour}{rgb}{0.95,0.95,0.95}

\lstdefinestyle{mystyle}{
    backgroundcolor=\color{backcolour},
    commentstyle=\color{codegreen},
    keywordstyle=\color{blue},
    numberstyle=\tiny\color{codegray},
    stringstyle=\color{codepurple},
    basicstyle=\ttfamily\footnotesize,
    breaklines=true,
    breakatwhitespace=true,
    numbers=left,
    numbersep=5pt,
    frame=single,
    tabsize=4
}

\lstset{style=mystyle, language=Matlab}

\usepackage{titlesec}

\titleformat{\chapter}
  {\normalfont\Large\bfseries} % format
  {Problem \thechapter:}       % prints "Problem n"
  {10pt}                       % spacing between number and title
  {\Large}                     % format for the title
  [\vspace{1ex}]               % optional vertical space after

\title{\textbf{Assignment 9}\\Numerical Methods: RK Method, Shooting Method and Boundary Value Problems}
\author{Sahil Raj\\\textit{2301CS41}}
\date{\today}

\begin{document}

\maketitle
\tableofcontents
\clearpage

\chapter{Numerical Solution of the Duffing Oscillator and Phase Space Plot using 4th Order Runge-Kutta}
\section*{Problem Statement}
The objective of this problem is to construct and analyze interpolating polynomials using the Lagrange interpolation method. Given a set of data points $(x_i, f(x_i))$, we approximate the function $f(x)$ by polynomials of varying order and compare their behavior. The interpolating polynomials are also used to estimate $f(7)$.

\begin{quote}
  \textbf{NOTE}: The code can be accessed using this link: \href{https://raw.githubusercontent.com/HavokSahil/computational-techniques-assignments/refs/heads/main/assignment3/a1.m}{MATLAB}, \href{https://raw.githubusercontent.com/HavokSahil/computational-techniques-assignments/refs/heads/main/assignment3/a1.jl}{Julia}.
\end{quote}


\section*{Methodology}
The Lagrange interpolation method constructs a polynomial $P_k(x)$ of degree $k$ that passes through $k+1$ data points. It is expressed as:
\[
  P_k(x) = \sum_{i=0}^{k} f(x_i) \, L_i(x),
\]
where the $i^{th}$ Lagrange basis polynomial is given by:
\[
  L_i(x) = \prod_{\substack{j=0 \\ j \neq i}}^{k} \frac{x - x_j}{x_i - x_j}.
\]

This ensures $L_i(x_j) = \delta_{ij}$, so that $P_k(x_j) = f(x_j)$.

\subsection*{Pseudo-code}
\begin{enumerate}
  \item Store the given data points $(X, Y)$.
  \item Define a function to compute the Lagrange basis $L_i(x)$ for a given order $k$.
  \item Define the interpolating polynomial function $P_k(x)$ using the basis functions and data points.
  \item For each order $k = 1, 2, 3$:
    \begin{itemize}
      \item Plot the interpolating polynomial along with the data points.
      \item Evaluate $P_k(7)$ and print the result.
    \end{itemize}
\end{enumerate}

\section*{Results}
The given dataset is:
\[
\begin{aligned}
(x, f(x)) = \{ &(0.5,\, 1.625), (1.5,\, 5.875), (3.0,\, 31.0), \\
               &(5.0,\, 131.0), (6.5,\, 282.125), (8.0,\, 521.0) \}.
\end{aligned}
\]

Interpolating polynomials of order $1, 2,$ and $3$ were constructed. The figure below shows the data points and the interpolated curves.

\begin{figure}[h!]
  \centering
  \includegraphics[width=1.0\textwidth]{a1.jpg}
  \caption{Interpolating Functions for Different Orders of Interpolation}
  \label{fig:a1}
\end{figure}

The computed estimates of $f(7)$ are:
\[
\begin{aligned}
  P_1(7) &= \, 29.250000 \\
  P_2(7) &= \, 208.000000 \\
  P_3(7) &= \, 351.000000
\end{aligned}
\]

\section*{Conclusion}
The Lagrange interpolation method was successfully applied to approximate the function given discrete data points. Lower-order polynomials (e.g., $k=1$) provide rough estimates and do not capture the curvature well, while higher-order polynomials fit the data more accurately. The evaluation of $f(7)$ demonstrates how increasing the order improves approximation quality. However, very high-order polynomials may suffer from oscillations (Runge’s phenomenon), so the choice of order must balance accuracy and stability.


\chapter{Temperature Evolution in a Rod: Solving the 1D Heat Equation}
\section*{Problem Statement}
The objective of this problem is to approximate the amplitude decay of a \textbf{damped harmonic oscillator} using rational interpolation. Given discrete amplitude measurements over time, the goal is to reconstruct the amplitude curve, evaluate it at specific time points, and visualize the interpolation.

\begin{quote}
  \textbf{NOTE}: The code can be accessed using this link: \href{https://raw.githubusercontent.com/HavokSahil/computational-techniques-assignments/refs/heads/main/assignment5/a2.m}{MATLAB}, \href{https://raw.githubusercontent.com/HavokSahil/computational-techniques-assignments/refs/heads/main/assignment5/a2.jl}{Julia}.
\end{quote}

\section*{Methodology}
The amplitude of a damped harmonic oscillator decreases over time. Given a set of discrete samples of amplitude versus time:
\[
(T_i, A_i) = \{(0, 10), (2, 5.5), (4, 3.5), (6, 2.6)\},
\]
we reconstruct a continuous approximation using \textbf{rational interpolation}.

\subsection*{Rational Interpolation}
1. \textbf{Reciprocal Difference Table:}
   - Construct the difference table \(D\) using a reciprocal-based scheme:
   \[
   D[i,1] = A_i, \quad D[i,j] = \frac{T_{i+j-1} - T_{j-1}}{D[i+1,j-1] - D[1,j-1]}.
   \]
2. \textbf{Rational Interpolant Construction:}
   - Using the extracted coefficients \(A\) from \(D\), the rational interpolant is recursively defined as:
   \[
   R(t) = A_1 + \frac{t - T_1}{A_2 + \frac{t - T_2}{\dots + \frac{t - T_{N-1}}{A_N}}}.
   \]

\subsection*{Steps}
\begin{enumerate}
    \item Construct the reciprocal difference table from the given amplitude samples.
    \item Extract rational coefficients for interpolation.
    \item Evaluate the interpolant over a fine time grid \(0:0.1:6.0\) s.
    \item Compute the amplitude at \(t = 5.0\) s as an example.
    \item Plot the interpolated curve along with the original data points for visualization.
\end{enumerate}

\section*{Results}
- The rational interpolant reconstructs the amplitude decay smoothly over the entire time interval.
- At \(t = 5.0\) s, the interpolated amplitude is:
\[
A(5.0 \text{ s}) \approx 2.9550
\]
- The plot shows that the interpolated curve passes closely through all measured data points, maintaining the expected decay behavior.

\begin{figure}[h!]
  \centering
  \includegraphics[width=0.93\textwidth]{a2.jpg}
  \caption{Rational interpolation of the damped harmonic oscillator amplitude (blue line) and original data points (red circles).}
\end{figure}

\section*{Conclusion}
Rational interpolation successfully reconstructs the amplitude decay curve from discrete measurements of a damped harmonic oscillator. The interpolated values match the trend of the original data, providing a smooth approximation and enabling estimation at intermediate time points. This demonstrates that rational interpolation is effective for modeling decaying or rapidly changing signals where polynomial interpolation might produce oscillations or inaccuracies.


\chapter{Numerical Solution of the 2D Poisson Equation on a Square Domain}
\section*{Problem Statement}
The purpose of this problem is to \textbf{numerically solve} the two-dimensional Poisson equation

\[
\frac{\partial^2 u}{\partial x^2} +
\frac{\partial^2 u}{\partial y^2}
= -2\pi^2 \sin(\pi x)\sin(\pi y)
\]

on the square domain:
\[
0 \le x \le 1,\qquad 0 \le y \le 1
\]

with boundary condition:
\[
u(x,y) = 0 \quad \text{for all boundaries}.
\]

The exact solution to this PDE is known:
\[
u(x,y) = \sin(\pi x)\sin(\pi y).
\]

This provides a good reference to verify the numerical method.

\section*{Methodology}

\subsection*{Finite Difference Discretization}

The domain is discretized using a uniform grid of size:
\[
N = 50,\qquad h = \frac{1}{N+1}.
\]

Using the second-order central difference approximation, the Poisson equation becomes:

\[
u_{i+1,j} + u_{i-1,j} + u_{i,j+1} + u_{i,j-1}
- 4 u_{i,j}
= -h^2 F_{i,j}
\]

where
\[
F(x,y) = -2\pi^2 \sin(\pi x)\sin(\pi y).
\]

Rearranging gives the classic \textbf{Jacobi iteration} formula:

\[
u_{i,j}^{(k+1)} =
\frac{1}{4}
\left(
u_{i+1,j}^{(k)} + u_{i-1,j}^{(k)} +
u_{i,j+1}^{(k)} + u_{i,j-1}^{(k)}
- h^2 F_{i,j}
\right).
\]

\subsection*{Implementation Steps}

\begin{enumerate}
    \item Construct a \( (N+2) \times (N+2) \) grid including boundaries.
    \item Initialize \(u = 0\) everywhere (consistent with boundary conditions).
    \item Compute the forcing term \(F(x,y)\).
    \item Iterate using Jacobi updates until:
          \[
          \| U^{(k+1)} - U^{(k)} \|_\infty < 10^{-6}
          \]
    \item Visualize the final solution using a surface plot.
\end{enumerate}

Convergence is checked through the maximum absolute difference between successive iterates.

\section*{Results}

The Jacobi method converged within the specified tolerance.  
The resulting numerical solution reproduces the expected shape of the analytic solution \(u(x,y) = \sin(\pi x)\sin(\pi y)\).

\begin{itemize}
    \item The solution exhibits a smooth peak at the center \( (0.5, 0.5 ) \).
    \item Boundary values correctly remain at zero.
    \item The surface plot clearly shows the symmetric sinusoidal structure.
\end{itemize}

\begin{figure}[h!]
    \centering
    \includegraphics[width=0.75\textwidth]{a3.jpg}
    \caption{Numerical solution of the 2D Poisson equation using the Jacobi iteration method.}
\end{figure}

\section*{Conclusion}

The Jacobi finite-difference method successfully solves the 2D Poisson equation on a square domain. The numerical solution matches the theoretical structure of the exact solution and demonstrates correct behavior under zero boundary conditions. Although Jacobi iteration converges slowly, it provides a simple and reliable method for validating PDE solvers.



\chapter{Time-Dependent Schrödinger Equation for a Harmonic Oscillator with Oscillating Potential}
\section*{Problem Statement}
The goal is to \textbf{numerically evaluate} the integral
\[
\int_{0}^{\infty} \frac{e^{-x}}{1 + x^2} \, dx
\]
using the \textbf{5-point Gauss-Laguerre quadrature} method.  
Gauss-Laguerre quadrature is particularly suited for integrals of the form
\(\int_{0}^{\infty} e^{-x} g(x) \, dx\).

\section*{Methodology}

\subsection*{Gauss-Laguerre Quadrature (5-point)}
For a function of the form
\[
\int_{0}^{\infty} e^{-x} f(x)\, dx,
\]
the 5-point Gauss-Laguerre quadrature approximates the integral as
\[
I \approx \sum_{i=0}^{4} w_i \, f(x_i),
\]
where \(x_i\) are the Laguerre nodes and \(w_i\) are the corresponding weights.

The 5-point nodes and weights used are:

\[
\begin{aligned}
x_0 &= 0.26356, & w_0 &= 0.521756 \\
x_1 &= 1.4134,  & w_1 &= 0.398667 \\
x_2 &= 3.59643, & w_2 &= 0.0759424 \\
x_3 &= 7.08581, & w_3 &= 0.00361176 \\
x_4 &= 12.6408, & w_4 &= 0.00002337
\end{aligned}
\]

\subsection*{Integrand}
The integrand is
\[
f(x) = \frac{1}{1 + x^2}.
\]

\subsection*{Implementation Steps}
\begin{enumerate}
    \item Store the 5 Gauss-Laguerre nodes and weights.
    \item Define the integrand \(f(x) = 1/(1+x^2)\).
    \item Compute the weighted sum:
    \[
    I = \sum_{i=0}^{4} w_i \, f(x_i)
    \]
    \item Display the final numerical value.
\end{enumerate}

\section*{Code}
\begin{lstlisting}[language=Matlab, caption={5-point Gauss-Laguerre Quadrature for $e^{-x}/(1+x^2)$}]
clc; clear all;

%% Program to compute the integral of fn exp(-x) / (1 + x^2)
% using 5-point Gauss-Laguerre technique
%
% Author: Sahil Raj
% Assignment 8 Problem 4

weights = [
  0.521756;
  0.398667;
  0.0759424;
  0.00361176;
  0.00002337;
];

nodes = [
  0.26356;
  1.4134;
  3.59643;
  7.08581;
  12.6408;
];

fn = @(x) 1/(1 + power(x, 2));

I = 0.0;
for i = 1:5
  I = I + weights(i) * fn(nodes(i));
end

fprintf("The value of the integral is: %f", I);
\end{lstlisting}

\section*{Results}
The numerical integration using 5-point Gauss-Laguerre quadrature gives:

\begin{itemize}
    \item \textbf{Integral Value:} \(I \approx 0.626379\)
\end{itemize}

\section*{Conclusion}
The 5-point Gauss-Laguerre quadrature provides an efficient and accurate method to evaluate
\[
\int_{0}^{\infty} \frac{e^{-x}}{1 + x^2} \, dx.
\]

For integrals of the type \(\int_{0}^{\infty} e^{-x} g(x) dx\), this method achieves high precision with very few nodes. The computed integral closely approximates the expected analytical value, demonstrating the effectiveness of Gauss-Laguerre quadrature for exponentially weighted integrals.




\appendix

\titleformat{\chapter}
  {\normalfont\Large\bfseries}
  {Appendix \thechapter:}
  {10pt}
  {\Large}
  [\vspace{1ex}]


\chapter{Octave Code for Problem 1}
\begin{lstlisting}
%% Program to solve for the Duffing Oscillator Equation using RK-4 Method
%
% Author: Sahil Raj
% Assignment 9 Problem 1

clc; clear all;

% CONSTANTS
k = -1;
b = 0.2;
l = 1;
F0 = 0.3;
w = 1.2;
m = 1.0;

% COEFFICIENTS
c1 = -k/m;
c2 = -l/m;
c3 = b/m;
c4 = F0/m;

% FUNCTION FOR SLOPE OF V(t)
F = @(x, v, t) c1*x + c2*x*x*x + c3*v + c4*cos(w*t);

x0 = 1.0;
v0 = 0.0;

% INITIAL VECTOR
Y0 = [
  x0;
  v0;
];

dt = 0.1;
tmin = 0.0;
tmax = 64.0;

ts = tmin:dt:tmax;
N = length(ts);

% SOLUTION ARRAY
Ys = zeros(N, 2);

% PLACE THE INITIAL CONDITION
Ys(1, :) = Y0;

for i = 2:N
  Y = Ys(i-1, :);
  x = Y(1);
  v = Y(2);
  t = tmin + (i-1) * dt;

  % slopes for x and v
  k1x = dt * v;
  k1v = dt * F(x, v, t);
  k2x = dt * (v + 0.5*k1v);
  k2v = dt * F(x + 0.5*k1x, v + 0.5*k1v, t + 0.5*dt);
  k3x = dt * (v + 0.5*k2v);
  k3v = dt * F(x + 0.5*k2x, v + 0.5*k2v, t + 0.5*dt);
  k4x = dt * (v + k3v);
  k4v = dt * F(x + k3x, v + k3v, t + dt);
  x_next = x + (k1x + 2*k2x + 2*k3x + k4x)/6;
  v_next = v + (k1v + 2*k2v + 2*k3v + k4v)/6;
  Ys(i,:) = [x_next, v_next];
endfor

% Extract results
Xs = Ys(:, 1);
Vs = Ys(:, 2);

% Create figure with 3 subplots
figure('Name','Duffing Oscillator Results','NumberTitle','off');

% Plot x(t)
subplot(3,1,1);
plot(ts, Xs, 'r', 'LineWidth', 1.5);
grid on;
xlabel('Time t');
ylabel('Displacement x(t)');
title('Duffing Oscillator: Displacement vs Time');

% Plot v(t)
subplot(3,1,2);
plot(ts, Vs, 'g', 'LineWidth', 1.5);
grid on;
xlabel('Time t');
ylabel('Velocity v(t)');
title('Duffing Oscillator: Velocity vs Time');

% Plot phase space (v vs x)
subplot(3,1,3);
plot(Xs, Vs, 'b', 'LineWidth', 1.5);
grid on;
xlabel('Displacement x(t)');
ylabel('Velocity v(t)');
title('Duffing Oscillator: Phase Space');
axis tight;

% Optional: add arrow markers along the phase trajectory to indicate direction
hold on;
quiver(Xs(1:10:end-1), Vs(1:10:end-1), ...
       diff(Xs(1:10:end)), diff(Vs(1:10:end)), 0.5, 'k');
hold off;

% Improve spacing between subplots
sgtitle('Duffing Oscillator Dynamics (RK4 Method)');

\end{lstlisting}

\chapter{Octave Code for Problem 2}
\begin{lstlisting}[language=Matlab,basicstyle=\ttfamily\small,breaklines=true]
%% Program to compute the definite integral of the function using
%% Gauss Quadrature Technique (4-points)
%%
%% Author: Sahil Raj
%% Assignment 7 Problem 2

%% For function 1/x+1 in range [0, 1] using change of variables

clc; clear all;

% Precomputed Node points from the table
x0 = -0.86114;
x1 = -0.33998;
x2 =  0.33998;
x3 =  0.86114;

% Precomputed weights from the table
w0 = 0.34785;
w1 = 0.65215;
w2 = 0.65215;
w3 = 0.34785;

function y = f(x)
   y = 1 / (x+3);
endfunction

I = w0*f(x0) + w1*f(x1) + w2*f(x2) + w3*f(x3);

printf("Integral of the function in [-1, 1] is: %f\n", I);
\end{lstlisting}

\chapter{Octave Code for Problem 3}
\begin{lstlisting}[language=Matlab,basicstyle=\ttfamily\small,breaklines=true]
%% Program to compute the expansion of the given
%% function in terms of legendre polynomials
%%
%% Author: Sahil Raj
%% Assignment 7 Problem 3

clc; clear all;

% Load data
T = dlmread("Guitar\_Theta\_R.txt", "\t");
A = T(:,1);   % angles in degrees
D = T(:,2);   % distances

theta = deg2rad(A);
x = cos(theta);

% Gauss-Legendre quadrature nodes and weights (10-point)
xi = [
  -0.9739065285;
  -0.8650633666;
  -0.6794095682;
  -0.4333953941;
  -0.1488743389;
   0.1488743389;
   0.4333953941;
   0.6794095682;
   0.8650633666;
   0.9739065285
];
wi = [
   0.0666713443;
   0.1494513491;
   0.2190863625;
   0.2692667193;
   0.2955242247;
   0.2955242247;
   0.2692667193;
   0.2190863625;
   0.1494513491;
   0.0666713443
];

% Interpolation function R(x)
function Rval = computeR(xval, xdata, Ddata)
    % Interpolate using pchip for stability
    Rval = interp1(xdata, Ddata, xval, 'pchip', 'extrap');
endfunction

% Legendre polynomial using recurrence
function Pl = legendre\_numeric(l, x)
    if l == 0
        Pl = 1;
        return;
    elseif l == 1
        Pl = x;
        return;
    end
    P0 = 1;
    P1 = x;
    for n = 1:l-1
        Pn1 = ((2*n+1) * x .* P1 - n * P0)/(n+1);
        P0 = P1;
        P1 = Pn1;
    end
    Pl = P1;
endfunction

% Compute coefficients
maxL = 15;      % maximum Legendre order
C = zeros(maxL+1,1);
for l = 0:maxL
    sumval = 0.0;
    for i = 1:length(xi)
        Rval = computeR(xi(i), x, D);
        Pl = legendre\_numeric(l, xi(i));
        sumval = sumval + wi(i) * Rval * Pl;
    end
    C(l+1) = (2*l+1)/2 * sumval;   % normalization factor for P\_l
end

% Reconstruct R(theta) from Legendre expansion
theta\_plot = linspace(min(theta), max(theta), 300);
x\_plot = cos(theta\_plot);
R\_rec = zeros(size(x\_plot));
for l = 0:maxL
    R\_rec = R\_rec + C(l+1) * legendre\_numeric(l, x\_plot);
end

% Plot
figure;
plot(rad2deg(theta\_plot), R\_rec, 'r-', 'LineWidth', 2); hold on;
plot(A, D, 'bo');
xlabel('Angle (degrees)');
ylabel('R(\theta)');
title('Radial Function and Legendre Expansion Approximation');
legend('Reconstructed R(\theta)','Original Data');
grid on;

\end{lstlisting}

\chapter{Octave Code for Problem 4}
\begin{lstlisting}[language=Matlab,basicstyle=\ttfamily\small,breaklines=true]
clc; clear all;
%%  Verification of the Orthogonality of Legendre Polynomials
%%  Using 3-Point Gauss-Legendre Quadrature
%%
%%  The 3-point quadrature formula is exact for polynomials of
%%  degree <= 5. Therefore, the integral of P\_i * P\_j is reliable
%%  only when i + j <= 5.
%%
%%  Author : Sahil Raj
%%  Course : Assignment 7 - Problem 4

N = 5;                 % Maximum order to test

%%  Quadrature Data (3-Point Gauss-Legendre)
%%  These weights and nodes integrate exactly up to degree 5.
global nP;    nP    = 3;
global lambda; lambda = [ 5/9; 8/9; 5/9 ];
global nodeps; nodeps = [ -sqrt(3/5); 0; sqrt(3/5) ];


%%  Custom Legendre Polynomials P\_0 ... P\_5
function y = LegendreCustom(n, x)
    switch n
        case 0, y = 1.0;
        case 1, y = x;
        case 2, y = (3*x*x - 1)/2;
        case 3, y = (5*x\^3 - 3*x)/2;
        case 4, y = (35*x\^4 - 30*x\^2 + 3)/8;
        case 5, y = (63*x\^5 - 70*x\^3 + 15*x)/8;
        otherwise
            error("LegendreCustom: order not implemented.");
    end
endfunction


%%  Gauss Quadrature Integrator
function I = gaussQuad(fn)
    global nP lambda nodeps;
    I = 0.0;
    for i = 1:nP
        I += lambda(i) * fn(nodeps(i));
    end
endfunction


%%  Compute Orthogonality Matrix
intmatrix = zeros(N+1, N+1);

for i = 0:N
    for j = 0:N
        fn = @(x) LegendreCustom(i,x) * LegendreCustom(j,x);
        I = gaussQuad(fn);

        % Apply normalization for diagonal entries:
        if i == j
            I *= (2*i + 1) / 2;
        endif

        intmatrix(i+1, j+1) = I;
    endfor
endfor

%% Display matrix
disp("Orthogonality Matrix:");
disp(intmatrix);


%%  PLOTTING SECTION

% Plot Legendre polynomials P0..P5
x = linspace(-1, 1, 400);
figure(1); hold on; grid on;
title("Legendre Polynomials P\_0 to P\_5");
xlabel("x"); ylabel("P\_n(x)");

for n = 0:5
    plot(x, arrayfun(@(t) LegendreCustom(n,t), x));
endfor
legend("P\_0","P\_1","P\_2","P\_3","P\_4","P\_5");


% Heatmap-like plot of orthogonality matrix
figure(2);
imagesc(intmatrix);
title("Orthogonality Matrix (Computed)");
xlabel("j"); ylabel("i");
colorbar();
set(gca, "XTick", 1:N+1, "YTick", 1:N+1);

\end{lstlisting}


\end{document}

