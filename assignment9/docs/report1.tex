\section*{Problem Statement}  
The objective of this problem is to \textbf{numerically solve} the \textbf{Duffing oscillator} equation:

\[
m \frac{d^2 x}{dt^2} + b \frac{dx}{dt} + k x + l x^3 = F_0 \cos(\omega t)
\]

with parameters: \(m=1\), \(k=-1\), \(b=0.2\), \(l=1\), \(F_0=0.3\), \(\omega=1.2\).  
The goal is to compute the time evolution of \(x(t)\) and \(v(t) = \frac{dx}{dt}\), and plot the phase space diagram using the \textbf{fourth-order Runge-Kutta (RK-4) method}.

\section*{Methodology}  

\subsection*{Fourth-Order Runge-Kutta Method}  
The Duffing oscillator is a second-order ODE. We rewrite it as a system of two first-order ODEs:

\[
\begin{aligned}
\frac{dx}{dt} &= v \\
\frac{dv}{dt} &= \frac{1}{m} \big( -k x - l x^3 - b v + F_0 \cos(\omega t) \big)
\end{aligned}
\]

The RK-4 method approximates the solution iteratively:

\[
\begin{aligned}
k_1 &= f(t_n, y_n) \, dt \\
k_2 &= f(t_n + \frac{dt}{2}, y_n + \frac{k_1}{2}) \, dt \\
k_3 &= f(t_n + \frac{dt}{2}, y_n + \frac{k_2}{2}) \, dt \\
k_4 &= f(t_n + dt, y_n + k_3) \, dt \\
y_{n+1} &= y_n + \frac{k_1 + 2 k_2 + 2 k_3 + k_4}{6}
\end{aligned}
\]

Here \(y_n = [x_n, v_n]^T\) and \(f\) represents the system of derivatives.

\subsection*{Implementation Steps}  
\begin{enumerate}
    \item Define the initial conditions: \(x(0) = 1\), \(v(0) = 0\).
    \item Specify constants: \(m, k, l, b, F_0, \omega\).
    \item Set time step \(\Delta t = 0.1\) and time range \(t \in [0, 64]\).
    \item Iterate the RK-4 scheme to compute \(x(t)\) and \(v(t)\) for each time step.
    \item Plot:
    \begin{itemize}
        \item \(x(t)\) vs \(t\)
        \item \(v(t)\) vs \(t\)
        \item Phase space: \(v(t)\) vs \(x(t)\)
    \end{itemize}
\end{enumerate}

\section*{Results}  
The numerical solution produces the following behavior:

\begin{itemize}
    \item The displacement \(x(t)\) shows a non-linear oscillatory pattern due to the cubic term.
    \item The velocity \(v(t)\) follows a corresponding non-linear trajectory.
    \item The phase space diagram exhibits a \textbf{limit cycle}, typical of a Duffing oscillator under periodic forcing.
\end{itemize}

\begin{figure}[h!]
    \centering
    \includegraphics[width=1.0\textwidth]{a1.jpg}
    \caption{Time evolution of \(x(t)\), \(v(t)\), and phase space diagram \(v\) vs \(x\) for the Duffing oscillator.}
\end{figure}

\section*{Conclusion}  
The 4th-order Runge-Kutta method accurately integrates the Duffing oscillator equation, capturing both the non-linear dynamics and the characteristic limit cycle in phase space.  
This method provides stable and precise results for moderately stiff non-linear ODEs with relatively large time steps.

